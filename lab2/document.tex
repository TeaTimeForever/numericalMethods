\documentclass{article}
\usepackage[utf8]{inputenc}
\usepackage[russian]{babel}
\usepackage{amsfonts}
\usepackage{natbib}
\usepackage{upquote}
\usepackage{datetime}
\usepackage{multicol}
\usepackage{listings}

\setlength{\voffset}{-2cm}
\setlength{\textheight}{700pt}

\title{Численные методы: Лабораторная работа №1}
\author{Группа 4001BV: Карина Пилюшонока \and Александр Степанов \and Борис
Кувшинников}
\date \today

\begin{document}

\maketitle
\newpage
\tableofcontents
\newpage
\section{Описание}
В данной лабораторной работе необходимо продемонстрировать работу алгоритма
интерполяции по принципy кубического сплайна. Также необходмо предоставить
график интерполирования на заданном промежутке.

В качестве исходных данных использовать следующий набор узлов:
\begin{table}[!h]
  \begin{tabular}{|l|l|l|l|l|l|l|l|}
  \hline
  \bfseries x & -3.2 & -2.1 & 0.4 & 0.7 & 2 & 2.5 & 2.777\\
  \hline
  \bfseries y &  10  & -2   & 0   & -7  & 7 & 0   & 0\\
  \hline
  \end{tabular}
\end{table}
\section{Ход работы}
Для вышеуказанных узлов в ходе работы алгоритма были получены следующие
коэффициенты для сплайнов:

\begin{table}[!h]
  \begin{tabular}{|l|l|l|l|l|}
  \hline
  \bfseries S & a & b & c & d \\
  \hline
  \bfseries 1 & 10& -3797.637713813412 & 0 & 3786.728622904321\\
  \bfseries 2 & -2& 7562.548154899551  & 10327.441698829965   & -33380.352401974465\\
  \bfseries 3 & 0 & 4596.569100537964  & 4298.791921190518   & 3786.728622904321\\
  \bfseries 4 & -7& -10955.52020307877 & 13258.938028365801 & -6270.33000302754\\
  \bfseries 5 & 7 & 320.23740655348223 & -1211.0542863131398 & 271.28973660308765\\
  \bfseries 6 & 0 & -76.94766995039464 & 416.68413330538596 & -38.47383497519732\\
  \hline
  \end{tabular}
\end{table}
    
\section{Выводы}

\section{Исходный код}
\begin{lstlisting}
var solveTridiagonal = require('./tridiagonal')
var p = [
  {x: 0, y: 2},
  {x: 1, y: 5},
  {x: 2, y: 3},
  {x: 3, y: 4}
]

var splines = []

//fill a
for (var i=0; i < p.length-1; i++) {
  splines[i] = {}
  splines[i].a = p[i].y
}

//fill h
for(var i=0; i < p.length-1;i++){
  splines[i].h = p[i+1].x - p[i].x
}

//fill matrix A and B ??
var mA = []
var mB = []
for(var i=1; i < p.length - 1; i++) {
  var eq = []
  if(i == 1) {
    eq.push(2 * (splines[i-1].h + splines[i].h))
    eq.push(splines[i].h)
  } else if(i == p.length - 2){
    eq.push(splines[i-1].h)
    eq.push(2 * (splines[i-1].h + splines[i].h))
  } else {
    eq.push[splines[i-1].h]
    eq.push(2 * (splines[i-1].h + splines[i].h))
    eq.push(splines[i].h)
  }
  mA.push(eq)
  mB.push([3 * ((p[i+1].y - p[i].y)/splines[i].h - (p[i].y - p[i-1].y)/splines[i-1].h)])
}
console.log("mA:" + mA)
console.log("mB:" + mB)

//fill c
var c = solveTridiagonal(mA, mB);
splines[0].c = 0
console.log(c)
for(var i=0; i < c.length; i++){
  splines[i+1].c = c[i]
}
splines.push({c: 0})

//fill d
for(var i=0; i<p.length-1; i++){
  console.log(i)
  splines[i].d = (splines[i+1].c - splines[i].c) / 3 * splines[i].h 
}

//fill b
for(var i=0; i<p.length-1; i++){
  splines[i].b = (p[i+1].y - p[i].y) / splines[i].h - splines[i].h/3 * (splines[i+1].c + 2*splines[i].c)  
}

function defineSIndex(x){
  for(var i=1; i < p.length; i++){
    if(x < p[i].x) return i-1
  }
  return i-1;
}

function interpolate(x){
  var sI = defineSIndex(x)
  var delta = x - p[sI].x
  return (splines[sI].a
         + splines[sI].b * (delta) 
         + (splines[sI].c * (Math.pow(delta, 2)))
         + (splines[sI].d * (Math.pow(delta, 3)))); 
}


for(var i=2; i <= 3; i+=0.1) {
  console.log("i: " + i + " = " + interpolate(i))
}

splines.pop()
console.log(splines)
\end{lstlisting}
\end{document}
