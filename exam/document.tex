\documentclass{article}
\usepackage[utf8]{inputenc}
\usepackage[russian]{babel}
\usepackage{amsfonts}
\usepackage{amsmath}
\usepackage{natbib}
\usepackage{upquote}
\usepackage{datetime}
\usepackage{multicol}
\usepackage{listings}
\usepackage{graphicx}
\usepackage{enumitem}

\begin{document}
\section{1. Представление чисел в памяти. Погрешности.}
Множ-во цел. чисел бесконечно, а проц. ограничен разрядной сеткой, кот-я
может оперировать только конечным подмножеством. Обычно выделяется 4байта что
позволяет представлять цел.числа в диапазоне $\mp 2 \cdot 10^9$


\end{document}
