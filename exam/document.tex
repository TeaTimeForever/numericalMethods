\documentclass{article}
\usepackage[utf8]{inputenc}
\usepackage[russian]{babel}
\usepackage{amsfonts}
\usepackage{amsmath}
\usepackage{natbib}
\usepackage{upquote}
\usepackage{datetime}
\usepackage{multicol}
\usepackage{listings}
\usepackage{graphicx}
\usepackage{enumitem}

\begin{document}
\section{Представление чисел в памяти. Погрешности.}
Мнж-во цел. чисел бесконечно, а проц. ограничен разрядной сеткой, кот-я
может оперировать только конечным подмножеством.
Обычно выделяется 4байта что позволяет представлять цел.числа в диапазоне $\pm 2
\cdot 10^9$.

В для решения задач в основном используют целые числа, в комп-е они
представляются как числа с плав. тчкй.
$\pm m \cdot 10^n$, где m-мантисса, 10-основание, n-порядок.
Нормализованная запись: $\pm 0.m \cdot 10^n$.
Число с фикс. точкой: -253.5
$x=\pm(\frac{d1}{\beta} + \frac{d2}{\beta^2} + \frac{dk}{\beta^k})\cdot \beta^n$
, где $\beta$ - основание системы, k - разрядность мантиссы, n - порядок, $d_i$
- числа от 0 до $\beta-1$, x - число.

Если я попробую записать число меньшее чем допустимо - запишется 0, если больше
чем допустимо, запишется самое большое допустимое число.

 \begin{table}[h!]
  \begin{tabular}{|l|l|l|l|}
  \hline
  \bfseries $\pm$ & мантисса & $\pm$ & порядок \\
  \hline
  \end{tabular}
\end{table}
Погрешность - мера точности для приближенных чисел. $x_0$ - точное знач., 
$\Delta x = |x_0 - x |$ - абсол. погр., $\delta x = \frac{\Delta x}{x_0}*100\%$
- относит. погр.\\
\begin{math}
\begin{array}{rl}
  \underline{a=a_0 \pm \Delta a} & 1) \Delta(a\pmb)= \Delta a + \Delta b\\
  \underline{b=b_0 \pm \Delta b} & 2) \delta(a \cdot or / b) = \delta a + \delta b \\
  3) \delta(a^n) = n*\delta a (мжн ум. погр if |a|<1) & 4) \Delta y = |f'(x_0)|\cdot \Delta x
\end{array}
\end{math}
Источники:модель,исх.данные, числ м-ды(интегр., ряды, табл.данн.),окргл. табл.
сетки.

\section{Обусловленность. Устойчивость и сходимость алгоритмов.}
Нек-е задачи очень чувствительны к неточностям в исх. данных - это устойчивость.
Пусть есть y(x), если исходн. величина $x$ - имеет погрешность $\Delta x$, то
результат $y$ имеет погрешность $\Delta y$: задача \textbf{устойчива} по
исх. пар-у $x$ если $y$ непрерывно от нее зависит, если малое приращение к
$\Delta x$ приводит к малому приращению $\Delta y$. О неустойчивых задачах
говорят, что они \textbf{чувствительны}, к входным данным.

Задача корректна, если для любых значений исходных данных, из некот-го класса,
ее решение существует, оно единственно и устойчиво.

Сходимость - близость получаемого решения задачи к истинному решению.\\
Сходимость итерационного процесса - способность итерационного алгоритма
достигать оптимума целевой функции, или подходить достаточно близко к нему, за
конечное число шагов. Скорость сходимости алгоритмов — один из важнейших
показателей качества аналитических моделей.

Под обусловленностью вычислительной задачи понимают чувствительность ее решения
к малым изменениям входных данных. Задачу называют хорошо обусловленной, если
при малых изменениях входных данных результат также изменяется незначительно.

\section{Прямые методы для ЛУ. Гаусс, Краут-Халецки, прогонка.}
Задача имеет 1 решение если: все у-я линейно независ., если кол-во уравнений =
кол-ву неизвестных. Прямые методы (подстановка, Крамер, Гаусс, Кр-Хл, прогонка.)

Гаусс - прямой ход (привидение матрицы к верхней \rhd) и обратный (вычисление
$x$ начиная с последней строки).\\
\begin{math}
\begin{array}{rl}
  a_1, b_1, c_1 = y_1 &\\
  a_2, b_2, c_2 = y_2 & I \cdot \frac{a_2}{a_1} + II =  \tilde{II}\\
  a_3, b_3, c_3 = y_3 & I \cdot \frac{a_3}{a_1} + III =  \tilde{III}
\end{array}
\end{math}
Гаусс с выбором ведущего элемента: если $|a_i| max $ в столбце - то перставляем
строку наверх (избавляет от деления на маленьк. числа).

Кр-Хл - представляем матр. $A$ как перемножение 2х матриц $A=C*D$, где С -
нижняя \rhd, а D - верхняя.
\begin{math}
\begin{array}{lr}
  c_{11}, _, _           & 1, d_{12}, d_{13}\\
  c_{21}, c_{22}, _      & _, 1, d_{23}  \\
  c_{31}, c_{32}, c_{33} & _, _, 1 \\
\end{array}
\end{math} 
Где $c_{i1} = a_{i1}$, $d_{1j} = a_{1j}/a{11}$ , 
$c_{ij} = a_{ij} - \sum\limits_{k=1}^{j-1} c_{ik} * d_{kj}$,
$d_{ij} = (a_{ij} - \sum\limits_{k=1}^{i-1} c_{ik} * d_{kj}) / c_{ii}$ \\
$AX = b -> CDX = B -> DX = Y and CY = B$ \\
$CY = B$ вычисляется как обратных ход Гаусса но сверху - узнали $Y$ \\
$DX = Y$ вычисляется как обратных ход Гаусса $X$. 

М-д прогонки - используется для 3х-диагональных матриц.
\begin{enumerate}
  \item $ x_1 
  = \underbrace{\frac{b_1}{a_{11}}}_\text{ $\beta 1$} 
  - \underbrace{\frac{a_{12} \cdot x_2}{a_{11}}}_\text{ $\alpha 1$} 
  = \beta 1 - \alpha 1 \cdot x_2$
  \item $a_{21}(\beta 1 - \alpha 1 \cdot x_2) + a_{22}x_2 + a_{23}x_3=b2$
  раскроем скобки $a_{21}\beta{1} - a_{21} \alpha{1} x_2 + a_{22}x_2 +
  a_{23}x_3=b2$, 
  $x_2(a_{22} - a{21}\alpha{1}) + a_{23}x_3 + a_{21}\beta{1} = b_2$,
  $x_2 = \underbrace{\frac{b_2 - a_{21}\beta{1}}{a_{22} - a_{21}\alpha{1}}}_\text{ $\beta 2$}
  - \underbrace{\frac{a_{23}}{a_{22} - a_{21}\alpha{1}}}_\text{ $\alpha 2$}\cdot x_3 
  = \beta{2} - \alpha{2} \cdot x_3$
  \item  $x_3 = \beta{3} -\alpha{3} \cdot x_4$\ldots
\end{enumerate}
$\beta{i} $- везде константа, $\alpha{i}$ - константа, говорит с каким коэф.
нужно брать $x_{i+1}$ чтобыполучить $x_{i}$.
$\beta{i} = \frac{b_i - a_{ii-1} \cdot \beta_{i-1}}{a_{ii} - a_{ii-1} \cdot \alpha_{k-1}}$
$\alpha{i} = \frac{a_{ii+1}}{a_{ii} - a_{ii-1} \cdot \alpha_{k-1}}$

\section{Обусловленность матрицы. Точность решения системы. Нормы вектора и матрицыю Способы оценки числа обусл. матрицы}
Величина  $Cond(A) = ||A|| \cdot ||A^-1||$ называется числом обусловленности
матрицы A. $Cond(A)$ определяет, насколько погрешность входных
данных может повлиять на решение системы; всегда $Cond(A) \geq 1$. Хотя число
обусловленности матрицы зависит от выбора нормы, если матрица хорошо обусловлена, то её число
обусловленности будет мало при любом выборе нормы.

Эвклидова норма вектора: $||x|| = \sqrt{x_1^2 = x_2^2 + x_3^2 + ..}$,
Манхеттенская норма вектора: $||x|| = |a_1| + |a_2| ..$,
Норма матрицы: максимальная норма одного из векторов матрицы.\\
1-теоретический $Cond(A) = = ||A|| \cdot ||A^-1||$\\
2-экспериментальный $\frac{X*}{X} \leq Cond(a) \cdot \frac{B*}{B}$, 
10< 1000 < 8

\section{Итерационные методы решения ЛУ}
М-д Якоби(метод простых итераций): $Ax = b$ -> $0 = b- Ax$ /*t -> 
$0=(b-Ax)t $ /+x -> x = $x=(b-Ax)t + x$ -> $x = tb - Axt + x$ ->
$x = (E - At)x + tb $ /B=E-At -> $x = Bx + tb$. Задаемся точностью.

М-д Гаусса Зейделя: выразим $x_1 x_2 x_3$ из \underline{соответствующих}
уравнений:\\
\begin{math}
\begin{array}{l}
  x_1^k = \frac{1}{a_{11}} (b1 - a_{12}x_2^{k-1} - a_{13}x_3^{k-1})\\
  x_2^k = \frac{1}{a_{22}} (b2 - a_{21}x_1^{k} - a_{23}x_3^{k-1})\\
  x_3^k = \frac{1}{a_{33}} (b3 - a_{31}x_1^{k} - a_{32}x_3^{k})
\end{array}
\end{math} 


\end{document}
