\documentclass{article}
\usepackage[utf8]{inputenc}
\usepackage[russian]{babel}
\usepackage{amsfonts}
\usepackage{amsmath}
\usepackage{natbib}
\usepackage{upquote}
\usepackage{datetime}
\usepackage{multicol}
\usepackage{listings}
\usepackage{graphicx}
\usepackage{enumitem}

\begin{document}
\section{Представление чисел в памяти. Погрешности.}
Мнж-во цел. чисел бесконечно, а проц. ограничен разрядной сеткой, кот-я
может оперировать только конечным подмножеством.
Обычно выделяется 4байта что позволяет представлять цел.числа в диапазоне $\pm 2
\cdot 10^9$.

В для решения задач в основном используют целые числа, в комп-е они
представляются как числа с плав. тчкй.
$\pm m \cdot 10^n$, где m-мантисса, 10-основание, n-порядок.
Нормализованная запись: $\pm 0.m \cdot 10^n$.
Число с фикс. точкой: -253.5
$x=\pm(\frac{d1}{\beta} + \frac{d2}{\beta^2} + \frac{dk}{\beta^k})\cdot \beta^n$
, где $\beta$ - основание системы, k - разрядность мантиссы, n - порядок, $d_i$
- числа от 0 до $\beta-1$, x - число.

Если я попробую записать число меньшее чем допустимо - запишется 0, если больше
чем допустимо, запишется самое большое допустимое число.

 \begin{table}[h!]
  \begin{tabular}{|l|l|l|l|}
  \hline
  \bfseries $\pm$ & мантисса & $\pm$ & порядок \\
  \hline
  \end{tabular}
\end{table}
Погрешность - мера точности для приближенных чисел. $x_0$ - точное знач., 
$\Delta x = |x_0 - x |$ - абсол. погр., $\delta x = \frac{\Delta x}{x_0}*100\%$
- относит. погр.\\
\begin{math}
\begin{array}{rl}
  \underline{a=a_0 \pm \Delta a} & 1) \Delta(a\pmb)= \Delta a + \Delta b\\
  \underline{b=b_0 \pm \Delta b} & 2) \delta(a \cdot or / b) = \delta a + \delta b \\
  3) \delta(a^n) = n*\delta a (мжн ум. погр if |a|<1) & 4) \Delta y = |f'(x_0)|\cdot \Delta x
\end{array}
\end{math}
Источники:модель,исх.данные, числ м-ды(интегр., ряды, табл.данн.),окргл. табл.
сетки.


\end{document}
