\documentclass{article}
\usepackage[utf8]{inputenc}
\usepackage[russian]{babel}
\usepackage{amsfonts}
\usepackage{natbib}
\usepackage{upquote}
\usepackage{datetime}
\usepackage{multicol}
\usepackage{listings}

\setlength{\voffset}{-2cm}
\setlength{\textheight}{700pt}

\title{Численные методы: Домашняя работа}
\author{Группа 4001BV: Карина Пилюшонока}
\date \today

\begin{document}

\maketitle
\newpage
\tableofcontents
\newpage
\section{Введение}
\subsection{Расчет индивидуальных коэффициентов для заданий}
Номер группы: 4001BV \\
Номер в журнале: 16 \\
Год посутпления: 2010 \\
Форма обучения: вечерняя


\begin{displaymath} 
  N_{g} = 3 * (4 + 2) + 0 - 1 = 17
\end{displaymath}

\begin{displaymath}
  N_{s} = 16
\end{displaymath}
\section{Метод исключения Гаусса}
\subsection{Подготовка: расчет уравнений для индивидуального задания}
\begin{displaymath}
\left(
  \begin{array}{ccc}
    (N_{g}+4)+5i & -3-4i & 4-4i \\
    -3+2i & 8+(10-N_{s})i & 1+2i \\
    (N_{g}+1)i & N_{s}-10 & N_{s}-N_{g}i
  \end{array}
\right)
=
\left(
  \begin{array}{ccc}
    3+6i\\
    1-(N_{s}-20)i\\
    10i
  \end{array}
\right)
\end{displaymath}
После вычисления отброса мнимой части из уравнений получаем следующие матрицы:
\begin{displaymath}
\left(
  \begin{array}{ccc}
    21 & -3 & 4 \\
    -3 & 8 & 1 \\
    0 & 6 & 16
  \end{array}
\right)
=
\left(
  \begin{array}{ccc}
    3\\
    1\\
    0
  \end{array}
\right)
\end{displaymath}
\end{document}
