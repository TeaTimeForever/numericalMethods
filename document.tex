\documentclass{article}
\usepackage[utf8]{inputenc}
\usepackage[russian]{babel}
\usepackage{amsfonts}
\usepackage{natbib}
\usepackage{upquote}
\usepackage{datetime}
\newcommand{\mR}{\mathbf{R}}
\newcommand{\mU}{\mathbf{U}}
\newcommand{\mS}{\mathbf{S}}

\title{Численные методы: Лабораторная работа №1}
\author{Группа 4001BV: Карина Пилюшонока \and Александр Степанов \and Борис
Кувшинников}
\date \today

\begin{document}

\maketitle
\newpage
\tableofcontents
\newpage
\section{Формулировка задания}

\subsection{Метод исключения Гаусса с ведущим элементом}
Написать программную реализацию алгоритма для решения системы линейных
уравнений методом исключения Гаусса с ведущим элементом.
Для проверки алгоритма использовать следующие системы:

\begin{equation}\label{1st}
\left\{ \begin{array}{ll}
x_{1} - 2x_{2} + x_{3} = 2\\
2x_{1} - 5x_{2} - x_{3} = -1\\
-7x_{1} + x_{3} = -2\\
\end{array} \right.
\end{equation}

\begin{equation}\label{2st}
\left\{ \begin{array}{ll}
5x_{1} - 5x_{2} - 3x_{3} + 4x_{4} = -11\\
x_{1} - 4x_{2} + 6x_{3} - 4x_{4} = -10\\
-2x_{1} - 5x_{2} + 4x_{3} - 5x_{4} = -12\\
-3x_{1} - 3x_{2} + 5x_{3} - 5x_{4} = 8\\
\end{array} \right.
\end{equation}

\begin{equation}\label{3d}
\left\{ \begin{array}{ll}
0.78x_{1} + 0.563x_{2} = 0.217\\
0.913x_{1} + 0.659x_{2} = 0.254\\
\end{array} \right.
\end{equation}

\subsection{Метод прогонки}
Написать программную реализацию алгоритма для решения системы линейных
уравнений методом прогонки для трехдиагональных матриц.
Для проверки алгоритма в списке линейных уравнений указанных в методическом
пособии для данной лабораторной работы не оказалось трехдиагональных матриц,
поэтому мы самостоятельно подготовили тестовые варианты для данной
части задания.

\subsection{Определение числа обусловленности матрицы}

Plain text.

\end{document}
