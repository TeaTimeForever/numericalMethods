\documentclass{article}
\usepackage[utf8]{inputenc}
\usepackage[russian]{babel}
\usepackage{amsfonts}
\usepackage{natbib}
\usepackage{upquote}
\usepackage{datetime}
\usepackage{multicol}

\title{Численные методы: Лабораторная работа №1}
\author{Группа 4001BV: Карина Пилюшонока \and Александр Степанов \and Борис
Кувшинников}
\date \today

\begin{document}

\maketitle
\newpage
\tableofcontents
\newpage
\section{Формулировка задания}

\subsection{Метод исключения Гаусса с ведущим элементом}
Написать программную реализацию алгоритма для решения системы линейных
уравнений методом исключения Гаусса с ведущим элементом.
Для проверки алгоритма использовать следующие системы:

\begin{displaymath}
  \left\{ \begin{array}{ll}
  x_{1} - 2x_{2} + x_{3} = 2\\
  2x_{1} - 5x_{2} - x_{3} = -1\\
  -7x_{1} + x_{3} = -2\\
\end{array} \right.
\end{displaymath}
\rule[1mm]{10cm}{0.1mm}
\begin{displaymath}
  \left\{ \begin{array}{ll}
  5x_{1} - 5x_{2} - 3x_{3} + 4x_{4} = -11\\
  x_{1} - 4x_{2} + 6x_{3} - 4x_{4} = -10\\
  -2x_{1} - 5x_{2} + 4x_{3} - 5x_{4} = -12\\
  -3x_{1} - 3x_{2} + 5x_{3} - 5x_{4} = 8\\
\end{array} \right.
\end{displaymath}
\rule[1mm]{10cm}{0.1mm}
\begin{displaymath}
  \left\{ \begin{array}{ll}
  0.78x_{1} + 0.563x_{2} = 0.217\\
  0.913x_{1} + 0.659x_{2} = 0.254\\
\end{array} \right.
\end{displaymath}

\subsection{Метод прогонки}
Написать программную реализацию алгоритма для решения системы линейных
уравнений методом прогонки для трехдиагональных матриц.
Также, необходимо сравнить метод прогонки и метод исключения Гаусса с выбором
ведущего элемента.
Для проверки алгоритма в списке линейных уравнений указанных
в методическом пособии для данной лабораторной работы не оказалось трехдиагональных матриц,
поэтому мы самостоятельно подготовили тестовые варианты для данной
части задания.
\begin{multicols}{2}
\begin{displaymath}
  \left\{ \begin{array}{ll}
  x_{1} + 4x_{2} = 17\\
  3x_{1} + 2x_{2} + 7x_{3} = 35\\
  x_{2} + 3x_{3} = 9\\
\end{array} \right.
\end{displaymath}

\begin{displaymath}
\left\{ \begin{array}{ll}
  x_{1} = 5\\
  x_{2} = 3\\
  x_{3} = 2\\
\end{array} \right.
\end{displaymath}
\end{multicols}
\rule[1mm]{10cm}{0.1mm}
\begin{multicols}{2}
\begin{displaymath}
  \left\{ \begin{array}{ll}
  0.2x_{1} + 1.2x_{2} = 3.42\\
  5.6x_{1} - 0.7x_{2} + x_{3} = 10.62\\
  -0.5x_{2} + 1.1x_{3} + 0.07x_{4} = -0.636\\
  1.9x_{3} + 1.3x_{4} = 4.92\\
\end{array} \right.
\end{displaymath}

\begin{displaymath}
\left\{ \begin{array}{ll}
  x_{1} = 1.5\\
  x_{2} = 2.6\\
  x_{3} = 0.4\\
  x_{4} = 3.2\\
\end{array} \right.
\end{displaymath}
\end{multicols}

\subsection{Определение числа обусловленности матрицы}
Написать программную реализацию алгоритма, для вычисления значения числа
обусловленности для матриц. В промежуточных вычислениях используется Евклидовая
норма матрицы. Для проверки алгоритма использовать следующие системы:

\begin{displaymath}
  \left\{ \begin{array}{ll}
  2x_{1} - x_{2} - x_{3}= 5\\
  x_{1} + 3x_{2} - 2x_{3} = 7\\
  x_{1} + 2x_{2} + 3x_{3} = 10\\
\end{array} \right.
\end{displaymath}
\rule[1mm]{10cm}{0.1mm}
\begin{displaymath}
  \left\{ \begin{array}{ll}
  0.78x_{1} + 0.563x_{2} = 0.217\\
  0.913x_{1} + 0.659x_{2} = 0.254\\
\end{array} \right.
\end{displaymath}
Пояснить полученные результаты вычислений.

\section{Ход выполнения заданий}
\subsection{Метод исключения Гаусса с ведущим элементом}
Полученные результаты:
\begin{multicols}{2}
\begin{displaymath}
  \left\{ \begin{array}{ll}
  x_{1} - 2x_{2} + x_{3} = 2\\
  2x_{1} - 5x_{2} - x_{3} = -1\\
  -7x_{1} + x_{3} = -2\\
\end{array} \right.
\end{displaymath}

\begin{displaymath}
\left\{ \begin{array}{ll}
  x_{1} = 0.52\\
  x_{2} = 0.08\\
  x_{3} = 1.64\\
\end{array} \right.
\end{displaymath}
\end{multicols}
\rule[1mm]{10cm}{0.1mm}
\begin{multicols}{2}
\begin{displaymath}
  \left\{ \begin{array}{ll}
  5x_{1} - 5x_{2} - 3x_{3} + 4x_{4} = -11\\
  x_{1} - 4x_{2} + 6x_{3} - 4x_{4} = -10\\
  -2x_{1} - 5x_{2} + 4x_{3} - 5x_{4} = -12\\
  -3x_{1} - 3x_{2} + 5x_{3} - 5x_{4} = 8\\
\end{array} \right.
\end{displaymath}

\begin{displaymath}
\left\{ \begin{array}{ll}
  x_{1} = -12.8235\\
  x_{2} = -2.2941\\
  x_{3} = 11.7647\\
  x_{4} = 19.2353\\
\end{array} \right.
\end{displaymath}
\end{multicols}
\rule[1mm]{10cm}{0.1mm}
\begin{multicols}{2}
\begin{displaymath}
  \left\{ \begin{array}{ll}
  0.78x_{1} + 0.563x_{2} = 0.217\\
  0.913x_{1} + 0.659x_{2} = 0.254\\
\end{array} \right.
\end{displaymath}

\begin{displaymath}
\left\{ \begin{array}{ll}
  x_{1} = 1\\
  x_{2} = -1\\
\end{array} \right.
\end{displaymath}
\end{multicols}
\subsection{Метод прогонки}
Полученные результаты:
\begin{multicols}{2}
\begin{displaymath}
  \left\{ \begin{array}{ll}
  x_{1} + 4x_{2} = 17\\
  3x_{1} + 2x_{2} + 7x_{3} = 35\\
  x_{2} + 3x_{3} = 9\\
\end{array} \right.
\end{displaymath}

\begin{displaymath}
\left\{ \begin{array}{ll}
  x_{1} = 5\\
  x_{2} = 3\\
  x_{3} = 2\\
\end{array} \right.
\end{displaymath}
\end{multicols}
\rule[1mm]{10cm}{0.1mm}
\begin{multicols}{2}
\begin{displaymath}
  \left\{ \begin{array}{ll}
  0.2x_{1} + 1.2x_{2} = 3.42\\
  5.6x_{1} - 0.7x_{2} + x_{3} = 10.62\\
  -0.5x_{2} + 1.1x_{3} + 0.07x_{4} = -0.636\\
  1.9x_{3} + 1.3x_{4} = 4.92\\
\end{array} \right.
\end{displaymath}

\begin{displaymath}
\left\{ \begin{array}{ll}
  x_{1} = 1.5\\
  x_{2} = 2.6\\
  x_{3} = 0.4\\
  x_{4} = 3.2\\
\end{array} \right.
\end{displaymath}
\end{multicols}

\subsection{Определение числа обусловленности матрицы}
Полученные результаты:
\begin{multicols}{2}
\begin{displaymath}
  \left\{ \begin{array}{ll}
  2x_{1} - x_{2} - x_{3}= 5\\
  x_{1} + 3x_{2} - 2x_{3} = 7\\
  x_{1} + 2x_{2} + 3x_{3} = 10\\
\end{array} \right.
\end{displaymath}

\begin{displaymath}
  cond = 3.4236
\end{displaymath}
\end{multicols}
\rule[1mm]{10cm}{0.1mm}
\begin{multicols}{2}
\begin{displaymath}
  \left\{ \begin{array}{ll}
  0.78x_{1} + 0.563x_{2} = 0.217\\
  0.913x_{1} + 0.659x_{2} = 0.254\\
\end{array} \right.
\end{displaymath}

\begin{displaymath}
  cond = 2.1932e+6
\end{displaymath}
\end{multicols}
Число обусловленности второй матрицы многократно превышает число первой
, поскольку она содержить в себе маленькие дробные числа.
Это означает, что при работе с матрицами подобного типа, стоит уделить
большое внимание алгоритмам, которые будут ими оперировать.

 \section{Сравнительный анализ методов прогонки и исключения Гаусса с
выборов ведущего элемента}

\section{Программная реализация}
\subsection{Метод исключения Гаусса с ведущим элементом}
\subsection{Метод прогонки}
\subsection{Определение числа обусловленности матрицы}
\end{document}
